\section{Experience}

\cventry{2021 -- present}{Senior ML Engineer}{Snapp!}{Tehran, Iran}{}{
      \begin{itemize}[label=\textbullet]
            \item Conceptualized \textbf{matrix factorization} techniques like \textbf{ALS} to recommend speed for streets that don't have sufficient data which \textbf{improved coverage from one million shard streets to 3 millions}.
            \item Developed a benchmarking service with Golang to benchmark our routing engines and models and report the results online in \textbf{Grafana} dashboards.
                  This service benchmarks around \textit{90,000} rides per day and supports both \textbf{ReST} and \textbf{gRPC} calls.
            \item Gathered data from different sources like company central \textbf{ClickHouse} and other teams' databases,
                  discussed with product managers and other teams to understand the data well.
                  Created a data gathering pipeline and ran it periodically using \textbf{Airflow}. Collected over 30 million rides for 2 months.
            \item Cleaned the data using our knowledge from the data and columns declaring confidence on the rides' ATA (Actual Time of Arrival)
                  also used outlier removal algorithms like \textbf{isolation forest} to remove outliers. It reduced our data to 1/2.
            \item Implemented feature engineering on the data for example using time as a cyclic feature,
                  adding extra features like \textbf{Haversine distance}, adding an understanding
                  of traffic behavior to feature vector, discretized geometric features for some models and etc.
                  Extended our feature vector size from 4 to 11.
            \item Completed \textbf{EDA} on data and shared dataset of Tehran rides to 4 smaller shards, trained a model for each shard that could reduce models' size and increase their accuracy.
            \item Trained and tested more than 5 different models like \textbf{Random Forests} and \textbf{Fully Connected Neural Networks}. Used \textbf{Keras Tuner} to find best structure for NN models.
            \item Developed a complete pipeline to train Neural Network models with different structures using \textbf{Tensorflow} that outputs results on metrics found cooperatively with product and commercial like \textbf{R2} and \textbf{negative error share}. It saves the model, its \textbf{Tensorboard} information and etc.
            \item Launched Neural Network models using \textbf{Tensorflow server} and and \textbf{Random Forest} models using \textbf{Fast API} on \textbf{Kubernetes}.
            \item Included preprocessing layers in the Neural Network model to avoid needing any middleware for data preprocessing.
            \item Load tested models using \textbf{K6}. The NN model's p90 response time was around 10ms and Random Forest's p90 response time was around 30ms. Made sure of online benchmarking, monitoring, tracing etc.
            \item Clustered Iran cities \textbf{Hierarchically} from over 40 down to 10 which helped in reducing number of models.
            \item Reduced short rides (rides that are finished under 10 minutes) \textbf{MAPE (mean absolute percentage error)} by 6\% and Reduced total rides MAPE by 2\%.
      \end{itemize}
}

\vspace{.5cm}

\cventry{2020 -- 2021}{Senior ML Engineer}{Snapp!}{Tehran, Iran}{}{
      \vspace{.4cm}
      \begin{itemize}[label=\textbullet]
            \item Re-designed data pipeline and services to improve its performance. Replaced old \textbf{spark}-based solution for driver location gathering with Golang to handle 40K driver
                  locations per second instead of former 8K per second.
            \item Upgraded \textbf{Cassandra} cluster to handle 200k per second write ops instead of 73k per second
            \item Used \textbf{Apache beam} over \textbf{Spark} so we could have tests for our pipeline stages.
            \item Changed the structure of data gathering to data driven using \textbf{Kafka} as CMQ that handles over 80k messages per second
            \item Deployed data tools in data pipeline for example \textbf{Airflow} for data gathering and preprocessing, \textbf{AutoML} tools like \textbf{H2O} to reduce time in training and testing models, \textbf{Feast} as feature store etc
            \item Mentored interns and onboard them on projects. Helped team on interviews and hiring process
            \item used \textbf{ONNX} to increase inference speed by 10\%
            \item Contributed in team's 2022 3rd IEEE Intelligent Vehicles Symposium (IV)
      \end{itemize}
}

\vspace{0.5cm}

\cventry{2018 -- 2020}{ML Engineer}{Avidnet Technology}{Tehran, Iran}{}{
      \vspace{.4cm}
      \begin{itemize}[label=\textbullet]
            \item Used \textbf{Kalman filtering} to know where a person is.
            \item Predicted where the person should be in a time bucket and alert otherwise.
            \item Used \textbf{Tensorflow Lite} to run the model on mobile phones.
            \item Set up a \textbf{Kafka} pipeline to gather data from sensors using \textbf{Protobuf} and then stores them into our \textbf{DataLake} (which is set up using \textbf{Postgres}).
            \item Drew the map of the house using accumulated GPS points
            \item Made video calls automaticallywith emergency contacts in case of abnormal behavior
            \item Provided in-application video call solution using WebRTC based on Pion Framework in Golang
            \item Set up turn server on AWS and use ELBs to open UDP/TCP ports
            \item Handled 1K concurrent calls based on our distributed design
            \item Set up project on Google Cloud Platform (\textbf{GCP}) Compute Engine using \textbf{Terraform}.
      \end{itemize}
}
